\documentclass[11pt]{article}
\pagestyle{plain}
\usepackage[lmargin=1.in,rmargin=1.in,tmargin=1.in,bmargin=1.in]{geometry}

\usepackage{fancyhdr, lastpage}
\usepackage{amsmath,amsthm}
\usepackage{tikz}



\usepackage{enumerate}
\fancypagestyle{plain}
{
\lhead{}
\rhead{}
\lfoot{}
\rfoot{\small{Page\ \thepage}}
\renewcommand{\headrulewidth}{0pt}
}
\pagestyle{fancy}
% \headrulewidth -2pt
% \footrulewidth 0pt
\renewcommand{\headrulewidth}{0.5pt}
\renewcommand{\footrulewidth}{0pt}
\lhead{\small\em Math 487, Fall 2019}
\rhead{\small Instructor: Longfei Li}
\chead{\small\em Homework 2}
\cfoot{}
\lfoot{}
% \rfoot{\small{Page\ \thepage\ of\ \pageref{LastPage}}}
\rfoot{\small{Page\ \thepage}}


\usepackage{hyperref}
\hypersetup{
  colorlinks = true,
  urlcolor=blue
}


\usepackage{graphicx,amssymb}
\parindent=0pt



\begin{document}

%\begin{center}
%{\large\bf Name:  }
%\end{center}
This is homework is due on: {\color{red}10/11/2019} 


Please email your code to me at \href{mailto:longfei.li@louisiana.edu}{longfei.li@louisiana.edu} before due date. \\


\section{Define Functions}\label{sec:funcDef}
 
\begin{enumerate}
%%%%%%%%%%%%% problem 1 %%%%%%%%%%%%% 
\item Write a Pyton function that solves the linear system $Ax=b$, where $A$ is an $n\times n$ tridiangal matrix with $-2$ on the main diagnal and $1$ on the diagnals above and below the main diagnal, i.e.,
  $$
  \begin{bmatrix}
    -2 & 1 & 0 & 0 &\dots& 0  & 0&0 &0 \\
    1 & -2 & 1 & 0 &\dots& 0  & 0&0 &0\\
    0 & 1 & -2 & 1 &\dots&  0  & 0&0 &0\\
    &   &\ddots  & \ddots & \ddots & & &   \\
    &    &   &\ddots  & \ddots & \ddots & &    \\
    & & & &\ddots  & \ddots & \ddots &     \\
     0  & 0&0 &0 &\dots  & -1& -2 & 1 & 0 \\
     0  & 0&0 &0  &\dots& 0 & 1 & -2 & 1 \\
      0  & 0&0 &0  &\dots& 0 &  0 & 1 & -2 \\
    \end{bmatrix}_{n\times n}
  $$
  And the vector $b$ is given by
  $$
  b=\begin{bmatrix}
  0\\
  0\\
  \vdots\\
  0\\
  -n-1
  \end{bmatrix}_{n \times 1}
  $$

The starter code for this problem is given in $\texttt{funcDef.py: solveLinearSystem(n,debug=False)}$ 

%%%%%%%%%%%%% problem 2 %%%%%%%%%%%%% 
\item
Write a Pyton  function to interpolate data points (xdata,ydata) using
  both linear and cubic spline interpolations.  The starter code for this problem is given in $\texttt{funcDef.py}$. Please finish the definition for the function $\texttt{interpolation1d(xdata,ydata,debug=False)}$

%%%%%%%%%%%%% problem 3 %%%%%%%%%%%%% 
\item Write a Python function to interpolate data points (xdata,ydata,zdata) using
 both linear and cubic spline interpolations.
  The starter code for this problem is given in $\texttt{funcDef.py}$. Please finish the definition for the function $\texttt{interpolation2d(xdata,ydata,zdata,debug=False)}$


\end{enumerate}

\section{Test your functions}
Write a test script to test the functions you developed for Section~\ref{sec:funcDef}. We have the following requirment for your tests for each problem.
\begin{enumerate}
\item solve the linear system for $n=10,100,1000$, respectively.
\item Using the given function
  $
  f=\cos(2\pi x)
  $
  to create  11 equally spaced data points on the interval $[-1,1]$.    Plot the interpolation polynomials, data points, and the known function in one figure. Make sure you have the appropriate legend for each plot.
  \item  Consider the given function
  $
  f=\cos(2\pi x)\cos(2\pi y)
  $. Randomly generate $n$ points on the domain [-1,1]x[-1,1]. Consider the case $n=50,100,  1000$and plot the interpolation polynomials, and the known function with sampled data points  in one 3 by 3 subplots.The subplots panel should have the following layout:
  \begin{table}[h]
    \centering
    \begin{tabular}{|c|c|c|}
      \hline
      linear interp  $n=50$ & cubic interp  $n=50$ &  known function with  50 samples\\      \hline
      linear interp  $n=100$ & cubic interp  $n=100$ &  known function with  100 samples\\      \hline
            linear interp  $n=1000$ & cubic interp  $n=1000$ &  known  function with 1000 samples\\      \hline
    \end{tabular}
  \end{table}

\end{enumerate}

\section{Hints for Each Problem}
\begin{enumerate}
\item Look for examples in Lecture 4 to create the matrix. You might want to use the $\texttt{eye}$ method from numpy
\item Look for examples in Lecture 8.
  \item Look for examples in Lecture 9.
\end{enumerate}


\section{Files to Submit}
Please submit two python files: $\texttt{funcDef.py}$ and $\texttt{test.py}$.
Please do not change the function interface and the filenames of the starter codes.


\end{document}
