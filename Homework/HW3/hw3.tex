\documentclass[11pt]{article}
\pagestyle{plain}
\usepackage[lmargin=1.in,rmargin=1.in,tmargin=1.in,bmargin=1.in]{geometry}

\usepackage{fancyhdr, lastpage}
\usepackage{amsmath,amsthm}
\usepackage{tikz}



\usepackage{enumerate}
\fancypagestyle{plain}
{
\lhead{}
\rhead{}
\lfoot{}
\rfoot{\small{Page\ \thepage}}
\renewcommand{\headrulewidth}{0pt}
}
\pagestyle{fancy}
% \headrulewidth -2pt
% \footrulewidth 0pt
\renewcommand{\headrulewidth}{0.5pt}
\renewcommand{\footrulewidth}{0pt}
\lhead{\small\em Math 487, Fall 2019}
\rhead{\small Instructor: Longfei Li}
\chead{\small\em Homework 3}
\cfoot{}
\lfoot{}
% \rfoot{\small{Page\ \thepage\ of\ \pageref{LastPage}}}
\rfoot{\small{Page\ \thepage}}


\usepackage{hyperref}
\hypersetup{
  colorlinks = true,
  urlcolor=blue
}


\usepackage{graphicx,amssymb}
\parindent=0pt



\usepackage{listings}
\usepackage{color}

\definecolor{dkgreen}{rgb}{0,0.6,0}
\definecolor{gray}{rgb}{0.5,0.5,0.5}
\definecolor{mauve}{rgb}{0.58,0,0.82}
\lstset{frame=tb,
 % aboveskip=3mm,
  %belowskip=3mm,
  showstringspaces=false,
  columns=flexible,
  basicstyle={\small\ttfamily},
  numbers=none,
  numberstyle=\tiny\color{gray},
  keywordstyle=\color{blue},
  commentstyle=\color{dkgreen},
  stringstyle=\color{mauve},
  breaklines=true,
  breakatwhitespace=true,
  tabsize=3
}




\begin{document}

%\begin{center}
%{\large\bf Name:  }
%\end{center}
This is homework is due on: {\color{red}11/01/2019} 


Please email your code to me at \href{mailto:longfei.li@louisiana.edu}{longfei.li@louisiana.edu} before due date. \\

\section{Function Definitions}

 
\begin{enumerate}
%%%%%%%%%%%%% problem 1 %%%%%%%%%%%%% 
\item  Write a {python} function to compute 
$$
\int_a^bf(x)\,dx,
$$
using the following  numerical  methods: Middle Point Rule, Trapezoidal Rule and Simpsons Rule (c.f. {\bf Lecture 10} ).
Your python function should have the following declaration:


\begin{lstlisting}[language=python, caption={myIntegration},label=code:myCode]
# write a function that integrates a function f from a to b with n sample points using a given  method
def myIntegration(f,a,b,n,method):
    #Input:
    #           f: a function integrant
    #          a: lower limit of the integration
    #          b: upper limit of the integration
    #          n: number of data points to sample
    # method: specify one of the methods: "middlePoint", "trapezoidal", "simpsons"
    #Output:
    #    value: results of integrating f from a to b
    # your method can be one of the three methods:
    if method=="middlePoint":
        # implement middle point rule here
        
    elif method=="trapezoidal":
        # implement trapezoidal rule here
    elif method=="simpsons":
       # implement simpsons rule here
    else:
        print("Unknown method specifed. We only support: middlePoint, trapezoidal, simpsons")
        
     return value
\end{lstlisting}  

Your code should check  that the inputs are valid; namely, you need to check the following situations
\begin{itemize}
\item we must have $a<b$
\item $n$ must be a positive integer. (We need different  minimum numbers of data points for each method )
\end{itemize}

Test your code extensively to make sure it's working as expected by calling it to compute different examples.


\item  Write a python function to solve the following two-point boundary value problem using the centered finite difference method.  
$$ 
  \begin{cases}
    u''+u'=f,  & x\in(0,1),\\
    u(0)=g_1,\\
    u(1)=g_2,\\
    \end{cases}
  $$
  where $f(x)$ is a given function, and $g_1$ and  $g_2$ are two given numbers for the boundary conditions. (refer to the matlab demo code \texttt{twoPointProblem.m} for hint.)

Your python function should have the following declaration:
\begin{lstlisting}[language=python, caption={solveTwoPointProblem},label=code:myCode]
# write a function to solve the following  two-point boundary value problem:
# u''+u'=f with u(0)=g_1, u(1)=g_2
def solveTwoPointProblem(f,g1,g2,N):
    #Inputs:
    #       f: a given function for the RHS
    #     g1: value for the left boundary condition
    #     g2: value for the right boundary condition
    #      N: number of grid points to use for the computation
    #

    # step 1: create matrix for the system
    M=  # hint: M=D2+D1

    # step 2: creat RHS
    F=

    # Implement boundary conditions:
    # modify M:


    # modify F:

    # Solve the linear system: Mu = F to find the solution
    

    return u
\end{lstlisting}


Test your code by solving the following two cases:
\begin{itemize}
\item  $f=0, g_1=0, g_2=1$
\item $f=-\pi^2\cos(\pi x)-\pi\sin(\pi x), g_1=1, g_2=-1$  
\end{itemize}  
It is helpful to work out the exact solution for each case when testing your code.


\item Redo the previous problems using MATLAB; that is, write the following two MATLAB functions
\begin{lstlisting}[language=matlab, caption={matlab function for problem 1},label=code:myCode]
function value=myIntegration(f,a,b,n,method)
% your implementation goes here      
end
\end{lstlisting}


\begin{lstlisting}[language=matlab, caption={matlab function for problem 2},label=code:myCode]
function  u=solveTwoPointProblem(f,g1,g2,N)
% your implementation goes here      
end
\end{lstlisting}

Note that MATLAB requires the function to be put inside a file with the same name. So you need the following matlab files:
\begin{itemize}
\item \texttt{myIntegration.m}
  \item \texttt{solveTwoPointProblem.m}
\end{itemize}


%\lstinputlisting[language=python, caption={myIntegration},label=code:myCode]{funcDefs.py}

\end{enumerate}


\section{General Instructions}
You want to test your code extensively to make sure it's working for various cases that you can think of. It is a good practice to have a test script to run all your tests; but you don't submit your test script this time.

I have provided you a python starter code named ``funcDefs.py'' for you to put all your python functions. I do not provide starter code for MATLAB functions, but they should be similar as your python codes.


\section{Files to Submit}
Please do not change the function interface and the filenames of the starter codes. And please submit the following files for this homework:
\begin{itemize}
\item \texttt{funcDefs.py}
\item \texttt{myIntegration.m}
\item \texttt{solveTwoPointProblem.m}
  \end{itemize}


\end{document}
