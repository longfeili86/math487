\frame
{
\frametitle{Computer Memory}

Common Units:
\begin{itemize}
\item[] bit: \quad binary digit (either an 'on' as 1  or an 'off' as 0)
\item[] byte:\quad 8 bits
\item[] KB: \quad 1 KB (Kilobyte)  = 1024 bytes
\item[] MB:\quad 1 MB (Megabyte) = 1024 KB
\item[] GB:\quad 1 GB (Gigabyte) =1024 MB
\item[] TB:\quad 1 TB (Terabyte) =1024 GB
\item[] PB:\quad 1 PB (Petabyte) = 1024 TB
\item[] EB: \quad 1 EB (Exabyte) = 1024 PB
\item[] more units: \quad ZB (Zettabyte), YB (Yottabye), etc.
 \end{itemize}
\bigskip

\textcolor{gray}{\bf\large Storing A Big Matrix}\\
\medskip
Q:~How much disk space is required to store a $10, 000 \times 10, 000$ matrix of real numbers?

 A:~Assuming 8 bytes (64 bits) are used for  each value.  A $10, 000 \times 10, 000$ matrix has $10^8$ elements
  So this requires $8 \times 10^8$ bytes = 800 MB
  
}



\begin{frame}[fragile]

\frametitle{Computer Memory}

Memory is subdivided into bytes (8 bits). One byte can store   $2^8=256$ distinct numbers: 
\begin{table}[h]
\begin{center}
\begin{tabular}{ccc}
binary &  & decimal\\ 
\texttt{  00000000} &&       0\\
\texttt{  00000001} &&       1\\
\texttt{  00000010} &&       2\\
\texttt{  00000011} &&       3\\
 \vdots& & \vdots\\
\texttt{  11111101} &&     253\\
\texttt{  11111110} &&     254\\
\texttt{  11111111} &&     255\\

\end{tabular} 
\end{center}
\end{table}

Might also represent characters, colors, etc.

Usually programs involve integers and real numbers that require more than 1 byte to store.
Often 4 bytes (32 bits) or 8 bytes (64 bits) used for each.

\end{frame}

\frame
{
\frametitle{Computer Memory}
To store integers, need one bit for the sign ($+$ or $?$).  In one byte this would leave 7 bits for binary digits.

Two-complements representation used:
\begin{table}[h]
\begin{center}
\begin{tabular}{ccc}
binary &  & decimal\\ 
\texttt{  10000000} &&     -128\\
\texttt{  10000001} &&     -127\\
\texttt{  10000010} &&     -126\\
\vdots && \vdots\\
\texttt{  11111101} &&       -3\\
\texttt{  11111110} &&       -2\\
\texttt{  11111111} &&       -1\\
\texttt{  00000000} &&        0\\
\texttt{  00000001} &&        1\\
\texttt{  00000010} &&        2\\
\texttt{  00000011} &&        3\\
\vdots && \vdots\\
\texttt{  01111101} &&      125\\
\texttt{  01111110} &&      126\\
\texttt{  01111111} &&      127
\end{tabular} 
\end{center}
\end{table}


See MATLAB code: \texttt{oneByteNumbers.m}


}


\frame
{
\frametitle{Integers}
Integers are typically stored in 4 bytes (32 bits). Values between  $-2^{31}$ and $2^{31}-1$ can be stored.
In Python, larger integers can be stored and will automatically be stored using more bytes.


}



\begin{frame}[fragile]

\frametitle{Fixed Point  Notation}
$64$ bits for a real number but always assume $N$ bits in integer part and $M$ bits in fractional part.

\medskip
Analog in decimal arithmetic, e.g.,\\
5 digits for integer part and\\
6 digits in fractional part

\begin{verbatim}
        pi:         00003.141592
pi / 10000:	        00000.000314
pi * 10000:         31415.926535
\end{verbatim}

Not very good representation. 
\begin{itemize}
\item[$\bullet$]  Precision depends on size of number
\item[$\bullet$]   Often many wasted bits (leading 0s)
\item[$\bullet$]   Limited range; often scientific problems involve very large or small numbers
\end{itemize}


\end{frame}




\begin{frame}[fragile]

\frametitle{Floating Point  Notation}
\textcolor{gray}{\bf\large Scientific Representation}\\

\begin{itemize}
\item[$\bullet$] Decimal (base 10):\\
\texttt{0.2345e-18} = $\red{0.2345} \times 10^{\blue{-18}} = 0.0000000000000000002345$\\
\medskip
\red{Mantissa:  0.2345}\\
\blue{Exponent:  -18}\\
\item[$\bullet$] Binary (base 2):\\
Example: \red{Mantissa}: 0.101101, \blue{Exponent}: -11011 means:
\begin{align*}
0.101101 &= 1(2^{-1}) + 0(2^{-2}) + 1(2^{-3}) + 1(2^{-4}) + 0(2^{-5}) + 1(2^{-6}) \\
	        &= 0.703125 (\text{in base 10})\\
	        \\
-11011 &= -[1(2^{4}) + 1(2^{3}) + 0(2^{2}) + 1(2^{1}) + 1(2^{0})  ]\\
	        &=-27 (\text{in base 10})
\end{align*}
So the number  is
$$
 0.703125 \times 2^{-27} \approx 5.2386894822120667 \times 10^{-9}
$$
\end{itemize}


\end{frame}

\begin{frame}[fragile]

\frametitle{Floating Point  Notation}

Python float is 8 bytes with IEEE standard representation.
53 bits for mantissa and 11 bits for exponent (64 bits = 8 bytes). We can store 52 binary bits of precision.
$$
2^{-52} \approx 2.2 \times 10^{-16} 
$$
Roughly, 15 digits of precision

For example,
\begin{verbatim}
In [1]: from numpy import pi

In [2]: pi
Out[2]: 3.141592653589793

In [3]: 1000*pi
Out[3]: 3141.592653589793

In [4]: pi/1000
Out[4]: 0.0031415926535897933

\end{verbatim}

Note: the decimal point is moving (floating) as oppose to the fixed point representation.

Note:  storage and arithmetic is done in base 2 and converted to base 10 only for printing.
\end{frame}



