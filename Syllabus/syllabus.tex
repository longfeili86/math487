\documentclass[11pt]{article}
\usepackage{geometry}                % See geometry.pdf to learn the layout options. There are lots.
\geometry{letterpaper}                   % ... or a4paper or a5paper or ... 
%\geometry{landscape}                % Activate for for rotated page geometry
%\usepackage[parfill]{parskip}    % Activate to begin paragraphs with an empty line rather than an indent
\topmargin=-.5in
\voffset=-.5in
\textwidth=6.6in
\textheight=8.9in
\oddsidemargin=0in
\parindent=0 in

\usepackage{hyperref}
\hypersetup{
  colorlinks = true,
  urlcolor=blue
}

\usepackage{enumitem}


%\DeclareGraphicsRule{.tif}{png}{.png}{`convert #1 `dirname #1`/`basename #1 .tif`.png}
\pagestyle{plain}
%\title{MATH 250 Fall 2017 Syllabus\vspace{-.5in}}
%\author{The Author}
%\date{}                                           % Activate to display a given date or no date

\begin{document}
%\maketitle
\begin{center}
  {\bf\Large MATH487: Computational Mathematics} \\
  Policy and Syllabus for Fall 2020\\
  \end{center}
%\section{}
%\subsection{}
\begin{minipage}[t]{0.2\linewidth}
{\bf Instructor:}\\  
{\bf Email:}\\  
{\bf Lecture:} \\
{\bf Office Hours:}\\
{\bf Delivery:}\\
%{\bf Calculator:}
\end{minipage}
\begin{minipage}[t]{0.8\linewidth}
Longfei Li\\
\href{mailto:longfei.li@louisiana.edu}{\tt longfei.li@louisiana.edu}  (best way to contact me)\\
MWF 09:00 AM - 09:50 AM\\
MWF 11:00 AM - 12:00 PM or by appointment\\
Lectures and office hours will be hosted online using Zoom\\
\end{minipage}


{\bf Tentative Plans for remote delivery}:
 \setlist{nolistsep}
\begin{enumerate}[noitemsep,leftmargin=0.55cm]
\item  Lectures will be delivered via Zoom following the class schedule; attendance during the lecture time is not mandatory. Recordings will be posted online after the lecture, so you can access the contents at your convenience.
\item Office hours will be hosted at the scheduled times on Zoom. You can join the live meeting anytime during the office hours to ask me questions. 
\item Class announcements and supplemental materials for the class will be posted on Moodle.
\item Homework and projects will be posted on Moodle. You are required to submit your individual work as a single pdf file on Moodle before the deadline. 
\item Get the latest information about UL Lafayette's return-to-campus plan and our continuing response to the COVID-19 pandemic at \href{https://louisiana.edu/covid19}{\em https://louisiana.edu/covid19}.\\
\end{enumerate}


 

{\bf Course Description}: Topics include machine arithmetic, computer architecture (i.e., floating point arithmetic etc.) and memory hierarchies; introduction of  MATLAB, python and their applications in computational mathematics; data analysis and machine learning using TensorFlow (if time permits).\\

{\bf Learning Outcomes}:  Successful completion of the course should enable you to have a basic understanding of computational mathematics. In addition, you should be able to have a basic working knowledge of  computational tools (such as MATLAB and python) and have the ability to use these tools to solve simple computational problems. \\

{\bf References  (there is no required text):}
 \setlist{nolistsep}
 \begin{enumerate}[noitemsep,leftmargin=0.55cm]
\item  \href{https://www.amazon.com/Learning-MATLAB-Tobin-Driscoll/dp/0898716837}{\em Learning MATLAB},  Tobin A. Driscoll.   
\item  \href{https://www.amazon.com/Numerical-Python-Scientific-Applications-Matplotlib/dp/1484242459/ref=sr_1_fkmr0_1?keywords=Numerical+Python\%3A+Scientific+Computing+and+Data+Science+Applications+with+Numpy\%2C+SciPy+and+Matplotlib+2nd+ed.+Edition&qid=1565725702&s=gateway&sr=8-1-fkmr0}{\em Numerical Python: Scientific Computing and Data Science Applications with Numpy, SciPy and Matplotlib (2nd Edition)},  Robert Johansson.\\
\end{enumerate}



%{\bf Attendance:} Attendance is mandatory. I will take attendance at the beginning of randomly selected lectures.  Absense will be recorded unless a written documented excuse is provided. A 2\% penalty will be applied to your final grades if you have more than 3  absen ses.\\


%% Other relevant online resources are listed below.
%%  \setlist{nolistsep}
%% \begin{enumerate}[noitemsep,leftmargin=0.55cm]
%% \item \href{https://staff.washington.edu/rjl/classes/am583s2014/}{\em High Performance Scientific Computing (AMATH 483/583 course materials, Spring Quarter, 2014, University of Washington)},  Randall J. LeVeque.
%%  \item \href{https://docs.python.org/3/tutorial/index.html}{\em The Python Tutorial}, Guido van Rossum
%%    and the Python development team.\\
%%   \end{enumerate}

{\bf Softwares}:  We need  Moodle, MATLAB,   Python  and Zoom for this course.
The  Moodle page  is  used to post class announcement,   assignments and grades. Please check the class Moodle page regularly to stay updated.   MATLAB and Python are needed for coding. Class lectures will be  delivered remotely  using Zoom.
\\

{\bf Assignments}: Four  programming homework will be assigned. Homework consists 40\% of the final grade. There are two  projects (midterm and final); each consists 30\% of the final grade.  A typed  report with necessary codes, figures and explanations should be turned in for each project. No report will be accepted after due date  unless a written documented excuse is provided. All the  assignments will be posted on Moodle.  Each student is expected to work independently and submit the work to Moodle before the deadline.
\\




{\bf Grading Mistakes}: Please check your graded work when returned to you. If you believe there is a grading mistake, you can ask me for a review within  {\bf 24 hours}. Scores may be adjusted at my discretion.  No points will be adjusted under the following circumstances:  (i) the graded paper is returned to you more than 24 hours and (ii) the graded paper is modified by you. \\

{\bf Final Grading Weights}:
\begin{table}[h!]
\centering
\begin{tabular}{c|c|c|c||c}
Homework  & Midterm Project  & Final Projects  & Total\\
\hline
 40\% & 30\% &30\% & 100\%\\
 \end{tabular}
\end{table}

{\bf Grading Scale}: 
\begin{table}[hp!]
\centering 
\begin{tabular}{c|c|c|c|c|c}
  Grade         & A & B & C & D & F \\\hline
  Percentage & 90\% -- 100\% & 80\% -- 89 \%      &   70\% -- 79 \%   &     60\% -- 69 \%  & $<$60\% 
\end{tabular}
\end{table}

%{\bf Emergency Evacuation Procedure}: A map of this floor is posted near the elevator marking the evacuation route and the Designated Rescue Area. This is an area where emergency service personnel will go first to look for individuals who need assistance in exiting the building. Students who may need assistance should identify themselves to the teaching faculty.


{\bf Academic Dishonesty}: ``The University holds that all work for which a student will receive a grade or credit shall be an original contribution or shall be properly documented to indicate sources. Abrogation of this principle entails dishonesty, defeats the purpose of instruction, and undermines the high goals of the University.​ Plagiarism is a specific type of cheating. It occurs when a student claims originality for the ideas or words of another person, when the student presents as a new and original idea or product anything which in fact is derived from an existing work, or when the student makes use of any work or production already created by someone else without giving credit to the source.''\\

University of Louisiana at Lafayette students are responsible for knowing the information, policies and procedures outlined in the Code of Conduct.
Academic dishonest will be handled according to the University’s code of conduct found at
\href{https://studentrights.louisiana.edu/student-conduct/code-conduct}{https://studentrights.louisiana.edu/student-conduct/code-conduct}.\\

\newpage

%% {\bf Tentative Lecture Plan} (subject to change):
%%     %% \newcounter{lesson}
%%     %% \newcommand{\ls}{%
%%     %%     \stepcounter{lesson}%
%%     %%     \thelesson}
%%     \begin{table}[hp!]
%%       \footnotesize
%%    \centering 
%%    \begin{tabular}{cp{10cm}}
%%      \hline\hline
%%      Lecture &    \multicolumn{1}{c}{Topic} \\\hline
%%      1 -- 5 &  Basic tools commonly used by the computational math community (shell scripting, git \& github,  working habits for documentation and reproducibility of the results). Knowledge about machine arithmetic, computer architecture (i.e., floating point arithmetic etc.), and memory hierarchies.  \\
%%      6 -- 7 & Language issues, compiled vs. interpreted. Demo using MATLAB, Python and fortran90. \\
%%      8 -- 24 & Focusing on python. Introduction of numpy, scipy and matplotlib.  Using python to solve problems in  computational mathematics (numerical PDE, linear algebra, etc).  MATLAB is also discussed for comparison purpose.\\
%%      25 -- 35 & Statistical learning and introduction of R (linear, logistic regression, classification, etc.).\\
%%      36 -- 38 & Introduction of machine learning (ternsorflow in python and R).\\
%%      39 -- 41 & Introduction of parallel computation (MPI in python and C++).\\
%%   \hline
       
    
%%    \end{tabular}

%%     \end{table}


   %%  {\bf Installation List}: \\
   %%  Here is the list of software packages that  will be used for this course: matlab,
   %% python3, ipython3, python3-numpy, python3-scipy, python3-matplotlib, git, gfortran, openmpi, tensorflow, and  R.






\end{document}  
